\documentclass{article}

\usepackage[margin=2.5cm]{geometry}

\usepackage[spanish]{babel}
\usepackage[T1]{fontenc}

\usepackage{graphicx}
\usepackage{fancyhdr}
\usepackage{fancyvrb}

\usepackage{tcolorbox}

\usepackage{listings}
\usepackage{xcolor}

\definecolor{codegreen}{rgb}{.2,0.6,0}
\definecolor{codegray}{rgb}{0.5,0.5,0.5}
\definecolor{codepurple}{rgb}{0.58,0,0.82}
\definecolor{codeblue}{rgb}{0,0.4,0.82}
\definecolor{codeorange}{rgb}{0.94,0.34,0.0}
\definecolor{backcolour}{rgb}{0.95,0.95,0.92}
\definecolor{backcolourgray}{rgb}{0.92,0.92,0.92}
\definecolor{codewhite}{rgb}{1,1,1}

\lstdefinestyle{mystyle}{
    backgroundcolor=\color{backcolourgray},   
    commentstyle=\color{codegreen},
    keywordstyle=\color{codeblue},
    numberstyle=\tiny\color{black},
    stringstyle=\color{codeorange},
    basicstyle=\ttfamily\footnotesize,
    breakatwhitespace=false,         
    breaklines=true,                 
    captionpos=b,                    
    keepspaces=true,                 
    %numbers=left,                    
    numbersep=5pt,                  
    showspaces=false,                
    showstringspaces=false,
    showtabs=false,                  
    tabsize=2,
    extendedchars=true,
    frame=single
    %, basicstyle=\footnotesize
}
\lstset{style=mystyle}

\usepackage{hyperref}
\hypersetup{
    colorlinks=true,
    linkcolor=blue,
    filecolor=magenta,      
    urlcolor=cyan,
}

\pagestyle{fancy}
\fancyhf{}
\rhead{Sistemas Operativos. Práctica 3.}
\lhead{Pablo Cuesta Sierra y Álvaro Zamanillo Sáez}
\cfoot{\thepage}



\setlength{\parskip}{0.15cm}



\begin{document}

\title{Sistemas Operativos. Práctica 3.}
\author{Pablo Cuesta Sierra y Álvaro Zamanillo Sáez}
\date{}
\maketitle

\begin{tcolorbox}
\tableofcontents
\end{tcolorbox}

\section*{Memoria compratida}

\addcontentsline{toc}{subsection}{Ejercicio 1}
\subsection*{Ejercicio 1}

a) Primero (línea 2), se intenta crear un segmento de memoria compartida, con las opción \texttt{O\_EXCL}, si existe el segmento, \texttt{shm\_open} devuelva error ($-1$). Después (línea 3), se trata este error y, en caso de que el error se deba a que el segmento con ese nombre ya existía, se abre este segmento (esta vez sin usar la flag de crear). Si el error no era porque ya existiera o si al intentar abrir el segmento ya existente hay otro error, se llama a \texttt{perror} y termina el programa. 

b) Para forzar la inicialización (en la primera llamada), habría que quitar la opción \texttt{O\_EXCL} (para que si existe, no se devuelva error), y añadir la opción \texttt{O\_TRUNC}, con la que, si existe, se trunca a tamaño 0. De este modo, se forzaría la inicialización. 


\addcontentsline{toc}{subsection}{Ejercicio 2}
\subsection*{Ejercicio 2}

a) Para obtener el tamaño basta con:

\begin{lstlisting}[language=C]
    if (fstat(fd, &statbuf) != 0) {
        perror("fstat");
        exit(EXIT_FAILURE);
    } 
\end{lstlisting} 
Y tras esto, el tamaño se encuentra en \texttt{statbuf.st\_size}.

b) Para truncar a 5 bytes:
\begin{lstlisting}[language=C]
    if (ftruncate(fd, 5) != 0) {
        perror("ftruncate");
        exit(EXIT_FAILURE);
    }
\end{lstlisting}
 
El fichero resultante contiene: \fbox{\texttt{Test\textvisiblespace}}. Es decir, solamente los $5$ primeros caracteres.


\addcontentsline{toc}{subsection}{Ejercicio 3}
\subsection*{Ejercicio 3}

a) Cada vez que se ejecuta el programa, el counter incrementa su valor en $1$. Esto es porque el programa abre el fichero en caso de que ya exista, y varía el valor que ya existe. Si borramos el fichero antes de ejecutar, el fichero se crea de nuevo y tras la ejecución el valor vuelve a ser $1$.

b) El fichero es binario, por lo que las variables no aparecen con los valores en forma de caracteres que se puedan leer con un editor de texto.


\addcontentsline{toc}{subsection}{ejercicio 4}
\subsection*{ejercicio 4}

a) No, ya que de ese modo, ningún otro lector podría abrir el segmento de memoria compartida.

b) No, porque se supone que es el escritor el primero que accede a la memoria compartida, para escribir en ella, por lo que debe ser el que determine su tamaño.

c) \texttt{mmap} crea un mapeo del segmento de memoria compartida dentro de las direcciones virtuales del proceso, por tanto se puede manejar como si esta memoria perteneciera al heap del proceso, con más facilidad que escribir en un fichero.

d) Sí se puede, habría que escribir en el segmento de memoria compartida con \texttt{write} tratándolo como un descriptor de fichero.

\addcontentsline{toc}{subsection}{ejercicio 5}
\subsection*{ejercicio 5}

a) Se envían en orden creciente (1,2,3..) pero se reciben según la prioridad con la que fueron enviados. Y para mensajes de la misma prioridad, siguiendo un criterio FIFO (first in, first out).

b) Si se cambia por \texttt{O\_RDONLY} la llamada a \texttt{mq\_send} devuelve error y el proceso termina. Del mismo modo, si se cambia por \texttt{O\_WRONLY} el error ocurre al intentar recibir los mensajes.



\addcontentsline{toc}{subsection}{ejercicio 6}
\subsection*{ejercicio 6}


a) El emisor envía el mensaje y el receptor lo recibe e imprime por \textit{stdout}

b) El receptor se bloquea en la orden \texttt{mq\_receive} hasta que el emisor envía el mensaje.

c) Si se ejecuta primero el emisor, el receptor recibe el mensaje y lo imprime correctamente. Si se ejecuta primero el receptor, este no se bloquea hasta recibir el mensaje, sino que la función \texttt{mq\_receive} devuelve error el estar la cola de mensajes vacía y se imprime el mensaje por \textit{stderr}
(\texttt{"Error  receiving  message"\\n}). 

d) (Asumimos que se crea la cola de mensajes como bloqueante.) No, ya que si un receptor llega antes que otro, el segundo se quedará esperando al siguiente mensaje en caso de que solo haya uno; o recibirá el siguiente mensaje de la cola en caso de que haya más.




\addcontentsline{toc}{subsection}{ejercicio 7}
\subsection*{ejercicio 7}


\subsubsection{Memoria compartida}

El proceso \textit{stream-ui.c} tiene que inicializar el segmento de memoria por lo que crea e
e\subsubsection{Semáforos}


\subsubsection{Colas de instrucciones}

end{document}



























%%%%%%%%%%%%%
\begin{lstlisting}[language=C, texcl=true, numbers=left]
\end{lstlisting}
%%%%%%%%%%%%%%%%%%%%%%%%%%%%
\begin{Verbatim}[frame=single]
cosas
\end{Verbatim}
%%%%%%%%%%%%%%%%%%%%%%%%%%%%
\begin{figure}[h]
\caption{Search 8.}
\bigskip
\includegraphics[scale=0.4]{ser8}
\centering
\label{fig:ser8}
\end{figure}
%%%%%%%%%%%%%%%%%%%%%%%%%%%%%%%%%%%%%%%%%%%%
\usepackage{listings}
\usepackage{xcolor}

\definecolor{codegreen}{rgb}{0,0.6,0}
\definecolor{codegray}{rgb}{0.5,0.5,0.5}
\definecolor{codepurple}{rgb}{0.58,0,0.82}
\definecolor{backcolour}{rgb}{0.95,0.95,0.92}

\lstdefinestyle{mystyle}{
    backgroundcolor=\color{backcolour},   
    commentstyle=\color{codegreen},
    keywordstyle=\color{codepurple},
    numberstyle=\tiny\color{codegray},
    stringstyle=\color{magenta},
    basicstyle=\ttfamily\footnotesize,
    breakatwhitespace=false,         
    breaklines=true,                 
    captionpos=b,                    
    keepspaces=true,                 
    numbers=left,                    
    numbersep=5pt,                  
    showspaces=false,                
    showstringspaces=false,
    showtabs=false,                  
    tabsize=2
}

\lstset{style=mystyle}
%%%%%%%%%%%%%%%%%%%%%%%%%%%%%%%%%%
\begin{lstlisting}[language=C]
bool findKey(const char * book_id, const char *indexName, int * nodeIDOrDataOffset){
    FILE *pf=NULL;
    int pos, comp;
    Node node;
    bool found = false;
    char b_id[5], search[5];

    if(!indexName)
        return false;

    if((pf=fopen(indexName,"rb"))==NULL)
        return false;

    memcpy(search, book_id, PK_SIZE);
    search[PK_SIZE] = '\0';
    /*read the root's position*/
    fread(&pos, sizeof(int), 1, pf);

    if(pos==-1){/*empty*/
        *nodeIDOrDataOffset=-1;
    }

    while(pos != -1 && found == false){
        /*find the node's offset in the file*/
        fseek(pf, INDEX_HEADER_SIZE+pos*sizeof(Node), SEEK_SET);
        /*read the node*/
        fread(&node, sizeof(Node), 1, pf);
        memcpy(b_id, node.book_id, PK_SIZE);
        b_id[PK_SIZE] = '\0';
        /*compare the primary key*/
        comp=strcmp(search, b_id);
        
        if(comp < 0){
            if(node.left==-1)/*not found*/
                (*nodeIDOrDataOffset) = pos;
            pos = node.left;            
        }
        else if(comp > 0){
            if(node.right==-1)/*not found*/
                (*nodeIDOrDataOffset) = pos;
            pos = node.right;            
        }
        else{/*found*/
            (*nodeIDOrDataOffset) = node.offset;
            found = true;
        }
    }
    fclose(pf);
    return found;
}

\end{lstlisting}

