\documentclass{article}

\usepackage[margin=2.5cm]{geometry}

\usepackage[spanish]{babel}
\usepackage[T1]{fontenc}

\usepackage{graphicx}
\usepackage{fancyhdr}
\usepackage{fancyvrb}

\usepackage{tcolorbox}

\usepackage{listings}
\usepackage{xcolor}

\definecolor{codegreen}{rgb}{.2,0.6,0}
\definecolor{codegray}{rgb}{0.5,0.5,0.5}
\definecolor{codepurple}{rgb}{0.58,0,0.82}
\definecolor{codeblue}{rgb}{0,0.4,0.82}
\definecolor{codeorange}{rgb}{0.94,0.34,0.0}
\definecolor{backcolour}{rgb}{0.95,0.95,0.92}
\definecolor{backcolourgray}{rgb}{0.92,0.92,0.92}
\definecolor{codewhite}{rgb}{1,1,1}

\lstdefinestyle{mystyle}{
    backgroundcolor=\color{backcolourgray},   
    commentstyle=\color{codegreen},
    keywordstyle=\color{codeblue},
    numberstyle=\tiny\color{black},
    stringstyle=\color{codeorange},
    basicstyle=\ttfamily\footnotesize,
    breakatwhitespace=false,         
    breaklines=true,                 
    captionpos=b,                    
    keepspaces=true,                 
    %numbers=left,                    
    numbersep=5pt,                  
    showspaces=false,                
    showstringspaces=false,
    showtabs=false,                  
    tabsize=2,
    extendedchars=true,
    frame=single
    %, basicstyle=\footnotesize
}
\lstset{style=mystyle}

\usepackage{hyperref}
\hypersetup{
    colorlinks=true,
    linkcolor=blue,
    filecolor=magenta,      
    urlcolor=cyan,
}

\pagestyle{fancy}
\fancyhf{}
\rhead{Sistemas Operativos. Práctica 3.}
\lhead{Pablo Cuesta Sierra y Álvaro Zamanillo Sáez}
\cfoot{\thepage}



\setlength{\parskip}{0.15cm}



\begin{document}

\title{Sistemas Operativos. Práctica 3.}
\author{Pablo Cuesta Sierra y Álvaro Zamanillo Sáez}
\date{}
\maketitle

\begin{tcolorbox}
\tableofcontents
\end{tcolorbox}

\section*{Memoria compratida}

\addcontentsline{toc}{subsection}{Ejercicio 1}
\subsection*{Ejercicio 1}

a) Primero (línea 2), se intenta crear un segmento de memoria compartida, con las opciones seleccionadas de tal manera que si existe, \texttt{shm\_open} devuelva error ($-1$). Después (línea 3), se trata este error y, en caso de que el error se deba a que el segmento con ese nombre ya existía, se abre este segmento (esta vez sin usar la flag de crear). Si el error no era porque ya existiera o si al intentar abrir el segmento ya existente hay otro error, se llama a \texttt{perror} y termina el programa. 

b) Para forzar la inicialización (en la primera llamada), habría que quitar la opción \texttt{O\_EXCL} (para que si existe, no se devuelva error), y añadir la opción \texttt{O\_TRUNC}, con la que, si existe, se trunca a tamaño 0. De este modo, se forzaría la inicialización. 


\addcontentsline{toc}{subsection}{Ejercicio 2}
\subsection*{Ejercicio 2}

a) Para obtener el tamaño basta con:

\begin{lstlisting}[language=C]
    if (fstat(fd, &statbuf) != 0) {
        perror("fstat");
        exit(EXIT_FAILURE);
    } 
\end{lstlisting} 
Y tras esto, el tamaño se encuentra en \texttt{statbuf.st\_size}.

b) Para truncar a 5 bytes:
\begin{lstlisting}[language=C]
    if (ftruncate(fd, 5) != 0) {
        perror("ftruncate");
        exit(EXIT_FAILURE);
    }
\end{lstlisting}
 
El fichero resultante contiene: \fbox{\texttt{Test\textvisiblespace}}. Es decir, solamente $5$ caracteres.


\end{document}


%%%%%%%%%%%%%
\begin{lstlisting}[language=C, texcl=true, numbers=left]
\end{lstlisting}
%%%%%%%%%%%%%%%%%%%%%%%%%%%%
\begin{Verbatim}[frame=single]
cosas
\end{Verbatim}
%%%%%%%%%%%%%%%%%%%%%%%%%%%%
\begin{figure}[h]
\caption{Search 8.}
\bigskip
\includegraphics[scale=0.4]{ser8}
\centering
\label{fig:ser8}
\end{figure}
%%%%%%%%%%%%%%%%%%%%%%%%%%%%%%%%%%%%%%%%%%%%
\usepackage{listings}
\usepackage{xcolor}

\definecolor{codegreen}{rgb}{0,0.6,0}
\definecolor{codegray}{rgb}{0.5,0.5,0.5}
\definecolor{codepurple}{rgb}{0.58,0,0.82}
\definecolor{backcolour}{rgb}{0.95,0.95,0.92}

\lstdefinestyle{mystyle}{
    backgroundcolor=\color{backcolour},   
    commentstyle=\color{codegreen},
    keywordstyle=\color{codepurple},
    numberstyle=\tiny\color{codegray},
    stringstyle=\color{magenta},
    basicstyle=\ttfamily\footnotesize,
    breakatwhitespace=false,         
    breaklines=true,                 
    captionpos=b,                    
    keepspaces=true,                 
    numbers=left,                    
    numbersep=5pt,                  
    showspaces=false,                
    showstringspaces=false,
    showtabs=false,                  
    tabsize=2
}

\lstset{style=mystyle}
%%%%%%%%%%%%%%%%%%%%%%%%%%%%%%%%%%
\begin{lstlisting}[language=C]
bool findKey(const char * book_id, const char *indexName, int * nodeIDOrDataOffset){
    FILE *pf=NULL;
    int pos, comp;
    Node node;
    bool found = false;
    char b_id[5], search[5];

    if(!indexName)
        return false;

    if((pf=fopen(indexName,"rb"))==NULL)
        return false;

    memcpy(search, book_id, PK_SIZE);
    search[PK_SIZE] = '\0';
    /*read the root's position*/
    fread(&pos, sizeof(int), 1, pf);

    if(pos==-1){/*empty*/
        *nodeIDOrDataOffset=-1;
    }

    while(pos != -1 && found == false){
        /*find the node's offset in the file*/
        fseek(pf, INDEX_HEADER_SIZE+pos*sizeof(Node), SEEK_SET);
        /*read the node*/
        fread(&node, sizeof(Node), 1, pf);
        memcpy(b_id, node.book_id, PK_SIZE);
        b_id[PK_SIZE] = '\0';
        /*compare the primary key*/
        comp=strcmp(search, b_id);
        
        if(comp < 0){
            if(node.left==-1)/*not found*/
                (*nodeIDOrDataOffset) = pos;
            pos = node.left;            
        }
        else if(comp > 0){
            if(node.right==-1)/*not found*/
                (*nodeIDOrDataOffset) = pos;
            pos = node.right;            
        }
        else{/*found*/
            (*nodeIDOrDataOffset) = node.offset;
            found = true;
        }
    }
    fclose(pf);
    return found;
}

\end{lstlisting}

