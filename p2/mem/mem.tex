\documentclass{article}
\usepackage[utf8]{inputenc}


\usepackage{geometry}
\geometry{a4paper, margin=2.5cm}

\usepackage[spanish]{babel}
\usepackage{graphicx}
\usepackage{fancyhdr}
\usepackage{fancyvrb}

\usepackage{listings}
\usepackage{xcolor}

\definecolor{codegreen}{rgb}{.2,0.6,0}
\definecolor{codegray}{rgb}{0.5,0.5,0.5}
\definecolor{codepurple}{rgb}{0.58,0,0.82}
\definecolor{codeblue}{rgb}{0,0.4,0.82}
\definecolor{codeorange}{rgb}{0.94,0.34,0.0}
\definecolor{backcolour}{rgb}{0.95,0.95,0.92}
\definecolor{backcolourgray}{rgb}{0.92,0.92,0.92}
\definecolor{codewhite}{rgb}{1,1,1}

\lstdefinestyle{mystyle}{
    backgroundcolor=\color{backcolourgray},   
    commentstyle=\color{codegreen},
    keywordstyle=\color{codeblue},
    numberstyle=\tiny\color{codegray},
    stringstyle=\color{codeorange},
    basicstyle=\ttfamily\footnotesize,
    breakatwhitespace=false,         
    breaklines=true,                 
    captionpos=b,                    
    keepspaces=true,                 
    %numbers=left,                    
    numbersep=5pt,                  
    showspaces=false,                
    showstringspaces=false,
    showtabs=false,                  
    tabsize=2
    %, basicstyle=\footnotesize
}

\lstset{style=mystyle}

\usepackage{hyperref}
\hypersetup{
    colorlinks=true,
    linkcolor=blue,
    filecolor=magenta,      
    urlcolor=cyan,
}

\pagestyle{fancy}
\fancyhf{}
\rhead{Sistemas Operativos. Práctica 2.}
\lhead{Pablo Cuesta Sierra y Álvaro Zamanillo Sáez}
\cfoot{\thepage}


\title{Sistemas Operativos. Práctica 2.}
\author{Pablo Cuesta Sierra y Álvaro Zamanillo Sáez}
%\date{}





\begin{document}

\maketitle

%\tableofcontents

\addcontentsline{toc}{section}{Semana 1}
\section*{Semana 1}

\addcontentsline{toc}{subsection}{Ejercicio 1}
\subsection*{Ejercicio 1}
a) La lista de señales se obtiene con el comando: 

\begin{lstlisting}[language=bash]
$ kill -l
\end{lstlisting}

b) \texttt{SIGKILL} corresponde al número $9$, y \texttt{SIGSTOP}, al número $19$. En este caso lo podemos ver directamente usando: \texttt{kill -l <Nombre\_de\_la\_señal>}

\subsection*{Ejercicio 2}

a) Basta con añadir el siguiente código en el hueco señalado: 
\begin{lstlisting}[language=C]
    if(kill(pid, sig)!=0){
        perror("kill");
        exit(EXIT_FAILURE);
    }
\end{lstlisting}

b) Si mandamos la señal \texttt{SIGSTOP}, al intentar excribir en la terminal, el texto que se introduce por el teclado, no aparece. Cuando mandamos la señal \texttt{SIGCONT}, el texto que habíamos intentado insertar antes aparece de golpe y la terminal vuelve a la normalidad.
De hecho, si excribimos algún comando y presionamos enter después de haberle mandado la señal \texttt{SIGSTOP}, en cuanto se mande la señal \texttt{SIGCONT}, se ejecuta el mismo de inmediato.

\subsection*{Ejercicio 3}
a) La llamada a \texttt{sigaction} no supone que se ejecute la función \texttt{manejador}, sino que establece que esa es la función que este proceso ejecutará para manejar la interrupción causada por la señal indicada en el primer argumento de \texttt{sigaction}.

b) En este caso, la única señal que se bloquea durante la ejecución de \texttt{manejador} es la señal que ha provocado su ejecución: \texttt{SIGINT}. Esto se debe a que \texttt{act.sa\_mask} no contiene ninguna señal.

c) Lo primero que aparece en pantalla es el \texttt{printf} del bucle. Luego el proceso se bloquea en la orden \texttt{sleep(9999)}, y cuando se manda la señal \texttt{SIGINT} con el teclado, aparece el \texttt{printf} de la función \texttt{manejador}. Como el manejador no termina el proceso, éste continúa por la orden siguiente a la que ejecutó por última vez, es decir, vuelve a empezar el bucle, imprimiendo el \texttt{printf} del \texttt{while}.

d) Si modificamos el programa para que no capture \texttt{SIGINT}, cuando pulsemos \texttt{Ctrl}$+$\texttt{C}, terminará la ejecución. Esto se debe a que si un proceso recibe una señal y no la captura, se ejecutará el manejador por defecto, el cual, en general, termina el proceso.

e) Las señales \texttt{SIGKILL} y \texttt{SIGSTOP} no pueden ser capturadas, ya que sólo las puede manejar el núcleo. Si no fuera así, el sistema operativo no tendría posibilidad de parar un proceso que no responde. 

\end{document}


%%%%%%%%%%%%%
\begin{lstlisting}[language=C, texcl=true, numbers=left]
\end{lstlisting}
%%%%%%%%%%%%%%%%%%%%%%%%%%%%
\begin{Verbatim}[frame=single]
cosas
\end{Verbatim}
%%%%%%%%%%%%%%%%%%%%%%%%%%%%
\begin{figure}[h]
\caption{Search 8.}
\bigskip
\includegraphics[scale=0.4]{ser8}
\centering
\label{fig:ser8}
\end{figure}
%%%%%%%%%%%%%%%%%%%%%%%%%%%%%%%%%%%%%%%%%%%%
\usepackage{listings}
\usepackage{xcolor}

\definecolor{codegreen}{rgb}{0,0.6,0}
\definecolor{codegray}{rgb}{0.5,0.5,0.5}
\definecolor{codepurple}{rgb}{0.58,0,0.82}
\definecolor{backcolour}{rgb}{0.95,0.95,0.92}

\lstdefinestyle{mystyle}{
    backgroundcolor=\color{backcolour},   
    commentstyle=\color{codegreen},
    keywordstyle=\color{codepurple},
    numberstyle=\tiny\color{codegray},
    stringstyle=\color{magenta},
    basicstyle=\ttfamily\footnotesize,
    breakatwhitespace=false,         
    breaklines=true,                 
    captionpos=b,                    
    keepspaces=true,                 
    numbers=left,                    
    numbersep=5pt,                  
    showspaces=false,                
    showstringspaces=false,
    showtabs=false,                  
    tabsize=2
}

\lstset{style=mystyle}
%%%%%%%%%%%%%%%%%%%%%%%%%%%%%%%%%%
\begin{lstlisting}[language=C]
bool findKey(const char * book_id, const char *indexName, int * nodeIDOrDataOffset){
    FILE *pf=NULL;
    int pos, comp;
    Node node;
    bool found = false;
    char b_id[5], search[5];

    if(!indexName)
        return false;

    if((pf=fopen(indexName,"rb"))==NULL)
        return false;

    memcpy(search, book_id, PK_SIZE);
    search[PK_SIZE] = '\0';
    /*read the root's position*/
    fread(&pos, sizeof(int), 1, pf);

    if(pos==-1){/*empty*/
        *nodeIDOrDataOffset=-1;
    }

    while(pos != -1 && found == false){
        /*find the node's offset in the file*/
        fseek(pf, INDEX_HEADER_SIZE+pos*sizeof(Node), SEEK_SET);
        /*read the node*/
        fread(&node, sizeof(Node), 1, pf);
        memcpy(b_id, node.book_id, PK_SIZE);
        b_id[PK_SIZE] = '\0';
        /*compare the primary key*/
        comp=strcmp(search, b_id);
        
        if(comp < 0){
            if(node.left==-1)/*not found*/
                (*nodeIDOrDataOffset) = pos;
            pos = node.left;            
        }
        else if(comp > 0){
            if(node.right==-1)/*not found*/
                (*nodeIDOrDataOffset) = pos;
            pos = node.right;            
        }
        else{/*found*/
            (*nodeIDOrDataOffset) = node.offset;
            found = true;
        }
    }
    fclose(pf);
    return found;
}

\end{lstlisting}

